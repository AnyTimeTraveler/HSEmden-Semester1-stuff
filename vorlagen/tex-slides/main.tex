\documentclass[mathserif,9pt]{beamer}
\usepackage[T1]{fontenc}
\usepackage[utf8]{inputenc}
\usepackage[ngerman]{babel}		% German style for date, sections ...

%\usepackage{helvet}

% Teilweise nützeliche Pakete:
%\usepackage[USenglish,english,ngerman]{babel}
%\usepackage{fancyhdr}
%\usepackage{hyperref}
%\usepackage{graphicx}
%\usepackage{array}
%\usepackage{tikz}
%\usepackage{fancyvrb}
%\usepackage{ctable}
%\usepackage{longtable}
%\usepackage{hhline}
%\usepackage{varioref}
%\usepackage{wrapfig}
%\usepackage{subfigure}
%\usepackage{eurosym}
%\usepackage{hvfloat}
%\usepackage{tabularx}
%\usepackage{bibgerm}
%\usepackage{listings}
%\usepackage{color}

\usepackage{fbtechnik-slides-100917}	% custom beamer style file: HS Emden/Leer

\title[Kurztitel]{Langer Titel}
\subtitle{Präsentation ...}
\author{Max Mustermann}
\institute{FB Technik | Abteilung Elektrotechnik und Informatik}
\date{\today}

%\date[31. August 2010]{\vspace{-30pt}\\31. August 2010\\[10pt]
%{\scriptsize
%\begin{tabular}{ll}
%	Erstprüfer:	& Prof. Dr. ... \\
%	Zweitprüfer:& Prof. Dr. ... \\
%\end{tabular}\\[-0cm]}}

\begin{document}
	\titleframe		% custom title frame with background logo
	
	\begin{frame}{Inhalt}
		%\begin{eblock}
			%\vspace{-1em}
			\tableofcontents%[hideallsubsections]
		%\end{eblock}
	\end{frame}

	\section{Einleitung}
		\begin{frame}{Das Thema}
			%\framesubtitle{NANANA}
			\begin{block}{Das Thema}
				Das Thema lautet ,,...''\\
				Hier könnte Ihre Werbung stehen!
			\end{block}
		\end{frame}
	
		\subsection{Alternativen}
			\begin{frame}{Alternativen}
				\begin{block}{Erste}
					\begin{itemize}
						\item Erstes Argument
							\begin{itemize}
							\item uno
							\item due
							\end{itemize}
						\item Zweites Argument
					\end{itemize}
				\end{block}
				
				\begin{block}{Zweite}
					\begin{itemize}
						\item ...
						\item ...
					\end{itemize}
				\end{block}
				
			\end{frame}

		\subsection{Alternative A}
			\begin{frame}{Die Erste}
				\begin{block}{Die Erste}
					\begin{itemize}
						\item ...
						\item ...
					\end{itemize}
				\end{block}
			\end{frame}

		\subsection{Alternative B}
			\begin{frame}{Die Zweite}
				\begin{block}{Die Zweite}
					\begin{itemize}
						\item ...
						\item ...
					\end{itemize}
				\end{block}
			\end{frame}

	\section{These}
		\begin{frame}{Die These}
			\begin{block}{Die These}
				Die These lautet ,,...''
			\end{block}
		\end{frame}
			
	\section{Analyse}
		\begin{frame}{Die Analyse}
			\begin{eblock}
				Einfacher Textinhalt
			\end{eblock}
			
			\begin{eblock}
				\vspace{-4pt} % Ohne Text läßt ein Block zu viel Platz
				\begin{columns}
					\column{0.01\textwidth}
						% Linker Abstand
					\column{0.45\textwidth}
						\begin{wblock}
							Bild 1
						\end{wblock}
					\column{0.45\textwidth}
						\begin{wblock}
							Bild 2
						\end{wblock}
					\column{0.01\textwidth}
						% Rechter Abstand
				\end{columns}
			\end{eblock}
		\end{frame}
		
\frame{ \frametitle{Organisation}

Das Modul ''Automaten'' besteht aus der Vorlesung (2~SWS) sowie dem Praktikum (2~SWS).

  \begin{itemize}
  \item Die Vorlesung bereitet das Praktikum vor. Sobald ausreichend Vorwissen aufgebaut ist, werden die Übungsblätter für das Praktikum via EMI-Board\footnote{ \hyperlink{http://board.et-inf.fho-emden.de/}{http://board.et-inf.fho-emden.de/} } veröffentlicht.
  \item Es werden 8 Übungsblätter bereitgestellt, die jeweils ca. 1 Wochen nach Veröffentlichung im Praktikum diskutiert werden. Die Bekanntgabe der Termine für das jeweilige Praktikum erfolgt ebenfalls via EMI-Board.
  \item Die bearbeiteten Übungen sind am Tag {\bf vor} dem Praktikum abzugeben\footnote{Lösung in Papierform in Briefkasten vor S110 einwerfen}! %Um zur Klausur zugelassen zu werden ist die Abgabe aller bearbeiteten Übungen obligatorisch.
\item Die Übungen dürfen in Zweiergruppen bearbeitet werden. Die Anmeldung hierzu erfolgt ebenfalls via EMI-Board.
  \end{itemize}
		
} % EOFrame		
				
	\section{Fazit}
	
	\section*{}
	\begin{frame}{Ende}
		\begin{block}{Ende}
			Vielen Dank für Ihre Aufmerksamkeit	...
		\end{block}
	\end{frame}
	
	%\begin{frame}{Literatur}
	%	\footnotesize
	%	\nocite*
	%	\bibliography{literatur}
	%	\bibliographystyle{geralpha}
	%\end{frame}

\end{document}
