\chapter{Umsetzung}
\label{sec:Umsetzung}

\section{Motivation}

Dies Kapitel \dots

\chapter{Bewertung der Ergebnisse}
\label{sec:Bewertung}

\section{Motivation}

In Kapitel \ref{sec:Umsetzung} wurde der auf der Aufgabenstellung und den Erkenntnissen aus der Aufarbeitung zum Stand der Technik basierende Lösungsansatz und dessen Realisierung beschrieben. Bezogen auf unser Beispiel: Die Realisierung einer Web-Cam. 
Es sei angenommen, sie funktioniert. Nun stellt sich die Frage: Wie gut funktioniert sie? Wurden die in der Aufgabenstellung definierten Parameter durch die Realisierung erreicht? Wo liegen die Limitationen? Was funktioniert nicht? 
 
\section{Systemperformanz}
\label{sec:Performanz}
 
Dieses Kapitel gibt einen Einblick, unter welchen Parametern das Ergebnis getestet wurde und quantifiziert die Performanz des Systems. \\

\example Wie viele Bilder können von der Web-Cam pro Sekunde aufgenommen und verarbeitet werden? Gehen sporadisch Bilder verloren? Unter welchen Umgebungslichtbedingungen funktioniert die Schaferkennung - und unter welchen Bedingungen versagt diese? 

\begin{enumerate}
\item Testaufbau definieren
\item Messreihen durchführen
\item Messreihen analysieren
\end{enumerate}

Hierzu bietet sich die Darstellung der Ergebnisse durch eine Tabelle an, beispielhaft dargestellt durch Tabelle \ref{tab:Ergebnis}.

\begin{table}[htb]

\begin{tabular}{lcr}
  Spalte 1 & Spalte 2 & Spalte 3 \\
  1 & 2 & 3 \\
 \end{tabular}
 \caption{Testszenarios und Performanz}
\label{tab:Ergebnis}
\end{table}