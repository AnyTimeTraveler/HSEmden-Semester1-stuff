
\chapter{Anhang}

\section{Weiterführende Erläuterungen}

Inhalte, die nicht im direkten Fokus der Aufgabenstellung stehen, jedoch zur Ausarbeitung indirekt beigetragen haben oder zum besseren Verständnis der dargestellten Aussagen beitragen, finden im Anhang der Arbeit eine passende Position.

\section{Formalien}

Der Inhalt ist wichtiger als die Verpackung. Dieser Grundsatz gilt ebenfalls für eine Projekt- oder Abschlussarbeit. Dennoch gilt es einen gewissen Standard bei der Gestaltung der Ausarbeitung einzuhalten. Dies Dokument kann beim Aufbau und der Gestaltung als Vorlage dienen. Um eine Ingenieurwissenschaftliche Arbeit zu verfassen stehen die Standard Office-Produkte wie beispielsweise MS Word zur Verfügung. Word wurde jedoch nicht zum Verfassen wissenschaftlicher Arbeiten mit Formeln, Abbildungen und Referenzen konzipiert und dies macht sich im Laufe der Arbeit durch offensichtliche Unzulänglichkeiten wie ... schnell bemerkbar. Es bedarf sehr viel Aufwand und Zeit, bis ein Dokument annähernd so professionell gestaltet ist wie beispielsweise mit dem Textsetzprogramm \LaTeX, das zum Verfassen wissenschaftlicher Texte\footnote{\url{https://de.wikipedia.org/wiki/LaTeX}} geschaffen wurde. 

\section{Messdaten}

Im begrenzten Umfang ist es auch hilfreich, weiteres Datenmaterial der Arbeit hinzuzufügen, beispielsweise Tabellen, Messreihen, kleinere Skripte, etc.
