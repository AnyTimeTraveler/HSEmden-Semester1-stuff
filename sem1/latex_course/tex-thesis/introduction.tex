
\chapter{Einleitung}
\label{sec:Einleitung}

\section{Motivation}
\label{sec:Motivation}

Das Verfassen von Projekt- und Abschlussarbeiten ist ein wichtiger Bestandteil des Studiums, stellt einige der wenigen gegenständlich vorzeigbaren Arbeitsergebnisse des Studiums dar und ist auch deshalb, beispielsweise bei Bewerbungen, von besonderer Bedeutung. 

Der Kern einer Projekt- und Abschlussarbeit im Rahmen eines Studiums ist es, problembezogen Fragestellungen selbständig auf ingenieurwissenschaftlicher Grundlage zu erarbeiten. Dies Dokument soll Studierenden der Lehreinheit Elektrotechnik und Informatik hierbei als Hilfestellung dienen. Der Fokus liegt weniger in der formalen Gestaltung als vielmehr im Aufbau und den zu bearbeitenden Inhalten einer ingenieurwissenschaftlichen Arbeit.\\

Die einzelnen Themengebiete, beginnend bei der Formulierung der Aufgabenstellung bis hin zur Performanzanalyse und Zusammenfassung der Ergebnisse des Projekts, werden zur Veranschaulichung durchgängig anhand eines Beispielprojektes verdeutlicht: Die Realisierung einer Web-Cam.

Gegenstand des einleitenden Kapitels ist es, dem Leser einen Überblick über für die Arbeit relevante Grundlagen und verwandten Arbeiten zu geben. Folgende grundlegende Fragen sollten beantwortet werden:
\begin{itemize}
	\item Warum wird das Thema der Arbeit behandelt? 
	\item Wie lautet die genaue Aufgabendefinition?
	\item Welche Fragenstellungen werden in der Arbeit behandelt?	
	\item Wie ist die Arbeit strukturiert?  
\end{itemize}

Das Kapitel zur Einleitung dient dem Leser vorrangig als Entscheidungshilfe: Ist die Arbeit für mich überhaupt relevant? Werden für mich interessante Fragestellungen in der Arbeit behandelt? Sofern den Leser nur Teilergebnisse interessieren: Wo finde ich diese?

Die Motivation zur Projektarbeit kann vielfältig sein und sollte daher ausreichend begründet werden. Beispiele: Ein bestehendes Problem wurde durch den Stand der Technik bisher nicht oder nicht zufriedenstellend gelöst. Oder: Gegenstand der Arbeit ist eine besonders kostengünstige Lösung. Oder auch: Die Arbeit behandelt eine grundlegende Evaluation, um Möglichkeiten als auch Limitationen einer neuen Technologie aufzuzeigen.

\section{Aufgabenstellung}
\label{sec:Aufgabe}

In der Aufgabenstellung wird nun. Wichtig: keine Lösungen vorab definieren - dies ist Aufgabe von Kapitel \ref{sec:Grundlagen} und insbesondere Abschnitt \ref{sec:StandderTechnik}!

\section{Aufbau der Arbeit}
\label{sec:Aufbau}

Um dem Leser den logischen Aufbau und die Zusammenhänge einzelner Kapitel zu verdeutlichen ("roter Faden"), bieten sich beispielsweise Formulierungen an wie "Nachdem im zweiten Kapitel die grundlegenden Komponenten einer Web-Cam behandelt wurde, wird im Kapitel~\ref{sec:Umsetzung} die Realisierung eines Systems zur Videoüberwachung von Schafen mittels einer intelligenten Web-Cam beschrieben. Anschließend folgt die Auswertung der in Kapitel~\ref{sec:Bewertung} definierten Experimente zur Performanz des Systems".


