
\chapter{Grundlagen und Stand der Technik}
\label{sec:Grundlagen}

\section{Motivation}
Gegenstand dieses Kapitels ist es, dem Leser einen Überblick über für die Arbeit relevante Grundlagen und verwandten Arbeiten zu geben. Folgende grundlegende Fragen sollten beantwortet werden:
\begin{itemize}
	\item Welche artverwandten Projektarbeiten oder Produkte existieren?
	\item Wo ist der Mangel zum Stand der Technik? Warum ist es notwendig oder relevant, das Thema in der Projektarbeit zu behandeln? 
	\item Wie grenzt sich der grundlegende Lösungsansatz vom Stand der Technik ab?
	\item Existieren Vorarbeiten, auf denen aufgebaut wird?
\end{itemize}

Die Motivation zur Projektarbeit kann vielfältig sein und sollte daher ausreichend begründet werden. Beispiele: Ein bestehendes Problem wurde durch den Stand der Technik bisher nicht oder nicht zufriedenstellend gelöst. Oder: Gegenstand der Arbeit ist eine besonders kostengünstige Lösung. Oder auch: Die Arbeit behandelt eine grundlegende Evaluation, um Möglichkeiten als auch Limitationen einer neuen Technologie aufzuzeigen.

\section{Stand der Technik}
\label{sec:StandderTechnik}

Zum Stand der Technik oder Stand des Wissens gehören alle Veröffentlichungen, die sich auf den Kern ihres Projektes beziehen. Oftmals wird der Fokus zu weit gefasst und unnötigerweise allgemein etablierte Literatur wiedergegeben.

\example Der Titel der Arbeit verspricht dem Leser etwas über eine Web-Cam für Schäfer zu erfahren. 
Die Recherche sollte daher auf Web-Cams mit besonderen Eigenschaften fokussieren. Welche artverwandten Projektarbeiten oder Produkte existieren? Auf Teilaspekte, beispielsweise die Ethernet-Verbindung oder das Netzwerkprotokoll, sollte im Detail nur eingegangen werden, wenn dieser Aspekt für die vorliegende Arbeit und die folgenden Kapitel relevant ist. Interessanter wäre: Wie werden Schafe bisher auf der Weide überwacht? Wo liegen hierbei die Nachteile und welche Aspekte haben die vorliegende Arbeit beeinflusst?\\

Als Startpunkt für einen ersten Überblick bieten sich Standard Internet-Suchmaschinen an. Die Qualität der Ergebnisse hängt hierbei grundlegend von den verwendeten Schlüsselwörtern ab. 

Der erste Überblick hilft, weitere relevante Schlüsselwörter zu definieren. Mit diesen bietet sich eine gezieltere Suche in fachspezifischen Datenbanken an. Beispiele:

\begin{itemize}
\item IEEE Explore \url{http://ieeexplore.ieee.org}
\item Google Schoolar \url{https://scholar.google.de/}
\item CiteSeer \url{http://citeseer.ist.psu.edu}
\item arXiv \url{http://arxiv.org}
\item ...
\end{itemize}

Innerhalb der Hochschule verfügen Sie über einen kostenfreien Zugang zu Publikationen der IEEE - nutzen Sie diese qualitativ hochwertige Quelle! 

Bitte verwenden Sie zur Darlegung des Stands der Technik vorrangig {\emph nicht flüchtige Quellen}, beispielsweise Publikationen in Form von Buch-, Zeitungs-, oder Konferenzbeiträgen.
Vermeiden Sie nach Möglichkeit {\emph flüchtige Quellen}, beispielsweise Informationen, die ausschließlich durch Internetseiten, Blog-Einträge und dergleichen belegt sind. Der Leser muss die Möglichkeit haben, die durch Literaturangaben hinterfütterten Aussagen noch nach Jahren nachvollziehen zu können - dies ist durch reiche Verwendung sekündlich veränderbarer Inhalte einzelner Internetseiten nicht gegeben.

Aber auch bei flüchtigen Internet-Quellen lassen sich in der Regel folgende Standard-Informationen zur Referenz angeben: Name des oder der Autoren, Titel der Veröffentlichung, Art der Veröffentlichung, Datum, Name der veröffentlichenden Institution.

\example Grundlegende Mechanismen zum Entwurf einer Web-Cam für Schäfer werden im Standard-Werk zum Thema Rechnerarchitekturen von Patterson und Hennessy~\cite{patterson2005} erläutert. 

Und auch Kameras, durch Eingebettete Systeme um eine künstliche Intelligenz zur Erkennung von Personen oder eingelernter Objekte erweitert wurden, sind seit Jahren etabliert.
Das können handelsübliche Web-Cams sein, oder auch als "Smart Cameras" bezeichneten Systeme aus dem Bereich Automation oder Automobil, siehe Spinneker und Koch in~\cite{spinneker2014}.

Zwar wird sogar von Klitzke und Koch in~\cite{klitzke2016} eine Überwachung von Baustellen mittels Videosensorik und Bildverarbeitung zur Diebstahlerkennung beschrieben, diese basiert jedoch auf einer kalibrierten Stereo-Kamera, die für den mobilen Einsatz in einer Schafherde als zu empfindlich gegenüber Vibrationen anzusehen ist.
\dots

Aufgrund der offenkundigen Nachteile etablierter Web-Cams bezogen auf eine Erkennung von Schafen, hat diese Arbeit den Entwurf und die Realisierung einer kostengünstigen Videosensorik für Schäfer ("") zum Gegenstand.

