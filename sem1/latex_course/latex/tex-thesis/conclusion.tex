\chapter{Zusammenfassung und Ausblick}

In diesem Kapitel wird ein Resume zu den Ergebnissen der Arbeit gegeben.
Hierbei ist eine selbstkritische Darstellung angebracht. Was funktioniert? Wie gut funktioniert es: Wie kann die Performanz des Systems numerisch beschrieben werden? Konnte die Aufgabenstellung vollständig umgesetzt werden? Was funktioniert nicht? \\

\example Das realisierte Kamerasystem ist in der Lage, bis zu 90 Bilder in der Sekunde aufzunehmen, siehe Abschnitt~\ref{sec:Performanz}. Dies geht signifikant über die in der Aufgabenstellung in Abschnitt~\ref{sec:Aufgabe} geforderten 60 Bilder pro Sekunde hinaus. Aufgrund der limitierten Bandbreite des CAN-Bus, ist das Kamerasystem jedoch ohne Bildkompression lediglich in der Lage, 50 Bilder pro Sekunde an einen PC zu übertragen.\\

Insbesondere die Abgrenzung zu Themen, die explizit nicht behandelt wurden, sollten hervorgehoben werden und dienen als Vorlage für den Ausblick auf Folgearbeiten. \\

\example Die Realisierung einer Bildkompression als auch eine deutliche Reduzierung des Energieverbrauchs, ist Gegenstand weiterführender Arbeiten.
