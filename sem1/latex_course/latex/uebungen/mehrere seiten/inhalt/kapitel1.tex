Leerzeilen und Zeilenumbrüche in der Eingabe erzeugen nicht immer unbedingt Leerzeichen
und Zeilenumbrüche im Ausgabedokument, wie man zunächst vermuten möchte. Eine Folge von mehreren Leerzeichen wird als ein Leerzeichen angesehen. D.h., zwischen Wörtern können beliebig viele Leerzeichen stehen und trotzdem wird der Zwischenraum nicht größer. Auch ein
einzelner Zeilenumbruch ist nur ein Leerzeichen.


Mehrere Zeilenumbrüche in Folge leiten jedoch einen neuen Absatz ein. Der Abstand zwischen
Absätzen ist aber immer gleich, egal wie viele Zeilenumbrüche ihr Eingabedokument an dieser
Stelle hat.


Mit diesem Mechanismus können Sie folglich Ihren Text im Eingabefile nach Belieben um-
brechen. Doppelte Leerzeichen können sich ebenfalls nicht einschleichen. Probieren Sie das
Hinzufügen zusätzlicher Leerzeichen und Zeilenumbrüche einfach an diesem Beispieltext aus.